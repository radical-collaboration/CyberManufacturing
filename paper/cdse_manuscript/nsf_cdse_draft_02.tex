\documentclass[preprint,11pt,authoryear]{elsarticle}
\usepackage{amsmath}
\usepackage[font=sf, labelfont={sf,bf}, margin=0cm]{caption}
\usepackage[inline]{enumitem}
\usepackage{epstopdf}   
\usepackage[margin=1in]{geometry}
\usepackage{lineno}
\usepackage{placeins}
\usepackage{subcaption}
\usepackage[usenames,dvipsnames]{color}
\usepackage[usenames,dvipsnames,svgnames,table]{xcolor}
\usepackage{algorithm}
\usepackage{algpseudocode}
\usepackage{textcomp}
\usepackage{graphicx}
\usepackage{tabulary}
\usepackage{float}
\usepackage[utf8]{inputenc}
\usepackage{nomencl}
\usepackage[font=sf, labelfont={sf,bf}, margin=0cm]{caption}  % Margin similar to Anik
\usepackage{placeins}
\usepackage[margin=1in]{geometry}
\usepackage{hyperref}
\usepackage[english]{babel} 
\usepackage{multirow}
\usepackage{nomencl}
\makenomenclature
\definecolor{listinggray}{gray}{0.95}
\definecolor{darkgray}{gray}{0.7}
\definecolor{commentgreen}{rgb}{0, 0.4, 0}
\definecolor{darkblue}{rgb}{0, 0, 0.4}
\definecolor{middleblue}{rgb}{0, 0, 0.7}
\definecolor{darkred}{rgb}{0.4, 0, 0}
\definecolor{brown}{rgb}{0.5, 0.5, 0}

\usepackage[normalem]{ulem}
\makeatletter
\def\cyanuwave{\bgroup \markoverwith{\lower3.5\p@\hbox{\sixly \textcolor{cyan}{\char58}}}\ULon}
\def\reduwave{\bgroup \markoverwith{\lower3.5\p@\hbox{\sixly \textcolor{red}{\char58}}}\ULon}
\def\blueuwave{\bgroup \markoverwith{\lower3.5\p@\hbox{\sixly \textcolor{blue}{\char58}}}\ULon}
\font\sixly=lasy6 % does not re-load if already loaded, so no memory problem.
\makeatother

\newif\ifdraft
\drafttrue
\ifdraft
\usepackage{xcolor}
\definecolor{ocolor}{rgb}{1,0,0.4}
\newcommand{\terminology}[1]{ {\textcolor{red} {(Terminology used: \textbf{#1}) }}}
\newcommand{\jwave}[1]{ {\reduwave{#1}}}
\newcommand{\jhanote}[1]{ {\textcolor{red} { ***shantenu: #1 }}}
\newcommand{\csnote}[1]{ {\textcolor{blue} { ***chaitanya: #1 }}}
\definecolor{orange}{rgb}{1,.5,0}
\definecolor{dandelion}{cmyk}{0,0.29,0.84,0}
\newcommand{\gpnote}[1]{{\textcolor{green} {***giannis: #1}}}
\newcommand{\note}[1]{ {\textcolor{magenta} { ***Note: #1 }}}
\else
\newcommand{\terminology}[1]{}
\newcommand{\jwave}[1]{#1}
\newcommand{\jhanote}[1]{}
\newcommand{\csnote}[1]{}
\newcommand{\gpnote}[1]{}
\newcommand{\note}[1]{}
\fi

\journal{Chemical Engineering Research and Design}
\begin{document}
\begin{frontmatter}

\title{A parallel unidirectional coupled DEM-PBM model for the efficient solution and simulation of 
computationally  intensive systems.}
\author[add1]{Chaitanya Sampat\corref{cor1}}
\author[add1]{Franklin Bettencourt\corref{cor1}}
%\thanks{These two authors has equal contribution.}
\author[add1]{Yukteshwar Baranwal}
\author[add2]{Ioannis Paraskevakos}
\author[add1]{Anik Chaturbedi}
\author[add1]{Subhodh Karkala}
\author[add2]{Shantenu Jha}
\author[add1]{Rohit Ramachandran}
\author[add1]{Marianthi Ierapetritou\corref{cor2}}
\address[add1]{Department of Chemical and Biochemical Engineering, Rutgers, The State University of New
Jersey, Piscataway, NJ, USA-08854}
\address[add2]{Electrical and Computer Engineering, Rutgers, The State University of New Jersey, 
Piscataway, NJ, USA-08854}
\cortext[cor1]{Equal contribution by C. Sampat and F. Bettencourt}
\cortext[cor2]{Corresponding author}
\ead{marianth@soe.rutgers,edu}

\begin{abstract}
Particulate processes are prevalent in the pharmaceutical industry but, the physics underlying 
these processes are complicated and difficult to predict. Modelling such processes using particle-
scale data is very computationally heavy and slow, while using a bulk model is not a very accurate
assumption. A quicker and a more accurate way to model this system is to use the particle-scale data 
into a bulk model. In this work, a uni-directional multi-scale model was used to model the high shear wet granulation 
process. A multi-dimensional population balance model (PBM) was developed which 
consisted of a mechanistic kernel, which obtained the collision data from the discrete element 
modelling (DEM) simulation. The PBM code was developed on 
C++ and was run in parallel using MPI + OMP hybrid technique. The DEM simulations were performed on
LIGGGHTS, which runs in parallel using MPI. Speedup of about 14 times was obtained for the PBM 
simulations and around 12 for the DEM simulations. This coupling was performed using the radical 
pilot for scaling studies from 1 processor to 128 for the PBM and up to 256 cores for the DEM. Using 
this developed framework, the granulation process has been modelled accurately much faster than
established techniques currently being used.
\end{abstract}
\begin{keyword}
Population balance model \sep Granulation \sep Discrete element methods  \sep MPI and OpenMP 
\sep Pharmaceutical process design
\end{keyword}
\end{frontmatter}
\linenumbers

\section{Introduction \& Objectives} 
Half of all industrial chemical production is typicall performed via the processing of particulate systems
(\cite{seville1997}). These processes account for about 70\% of the products 
of industries like detergents, aerosols, fertilizers, and pharmaceuticals\citep{Litster2016}. 
These processes are widely popular as these particulate products have great advantages over 
liquid formulations such as better chemical and physical stability and reduced 
transportation cost. Despite the prevalence of these particulate processes, the underlying 
physics of these processes is poorly understood \citep{Rogers2013}. As a result industries 
that rely on these processes, especially pharmaceuticals, have to use expensive 
heuristic studies and inefficient operational protocols with high recycle ratios to meet strict 
regulatory standards (\cite{Ramachandran2009}). This can drive up the costs and delay the release of 
new products. What makes these processes so challenging to design is that there are none or few governing 
equations to accurately predict their behavior \citep{sen2013}.

Particulate processes are defined by chaotic micro-scale phenomena that result from the many 
particle-particle interactions inside these systems. These small scale phenomenon develop into 
complex bulk behavior of these processes. To successfully predict the bulk behavior of these 
systems, a model needs to capture the particle-particle interactions and emergent meso-scale phenomena. 
The discrete Element Method (DEM)\citep{Cundall1979} has proven to be an accurate way 
to  model a bulk material properties of particulate systems using micro-scale particle level 
interactions for instance pharmaceutical processes \citep{Hancock2011}. DEM uses Newton's equations of 
motion to model the forces on each particle in the system and it's interactions with the system 
geometry and other particles. This enables DEM to model the small scale phenomena that 
determines the bulk behavior of particulate system. Since DEM includes large amount of interactions
and force computations, it usually takes a large amount of time to run. Thus another more computationally 
efficient, but potentially less accurate technique, used to model these particulate processes is population 
balance model (PBM) \textbf{need to cite 1-2 papers}.

PBM takes into account the changes in internal or spatial particle porperties. This 
model lacks sensitivity to design parameters such as equipment geometry. It is semi-mechanistic in 
nature, meaning they use population averages and probability to capture bulk behavior but still 
seek to capture some of the micro-phenomena of particle-particle interactions using correlations or 
empirically developed kernels to approximate those interactions. Even though PBM is much faster 
than DEM, a detailed PBM can still take a significant amount of time to solve which prevents them 
from being used as widely as they could be in academia and industries \citep{Barrasso2013}. Due to the 
populating averaging and their semi-mechanistic kernels, highly detailed PBMs still can have 
difficulties in capturing the micro-scale phenomena that are crucial to predicting accurate dynamics of 
particulate processes. 

Thus, to complement the limitations of each of these modeling methods they are coupled to 
provide a more accurate model.  The typical work flow of such a PBM-DEM coupled model involves 
using a short DEM simulation to capture the particle-particle level interactions of the system and 
then this physics is fed into the PBM, so that the PBM can more accurately simulate bulk system 
behavior \citep{Goldschmidt2003} \citep{Reinhold2012}\citep{Barrasso2013}. This PBM-DEM 
method is more accurate than a PBM simulation alone but runs much faster than a complete DEM 
simulation. Despite the performance benefits of these coupled PBM-DEM models they still take too 
long to solve. In the past parallel computing has been used to speed up computationally intensive 
problems such as Molecular Dynamics (MD), Computational Fluid Dynamics (CFD), but in recent 
years  this technique has been applied to DEM and PBM simulations 
\citep{Bettencourt2017}\citep{Prakash2013a}\citep{Gunawan2008}.

In order to solve large computational problems, parallel computing can be used to distribute the 
problem across many CPUs which can work together to solve the problem more quickly. To solve 
problems in parallel, computing clusters are often used and with drop  in the prices of hardware, 
they are becoming popular. A cluster is essentially a collection of conventional computers 
connected together via a high speed network like Ethernet or Infiniband. The difficulty with 
implementing parallel computing to solve a computational problem is that it involves a lot of 
computer-computer communication and algorithm redesigning which needs to be done very 
carefully to ensure numerical consistency and with consideration of the cluster the program will be 
run on.

High performance computing (HPC) is the utilization of supercomputers and parallel processing
techniques for solving complicated computational problems. Some of the potential benefits of using
HPC in the pharmaceutical industry include really high accuracy for parameter estimation as detailed 
particle-scale simulations can be performed quickly. On-line modeling of continuous processes as 
discussed by \citep{Bettencourt2017}, as their computational time of the simulations was less than 
half of the actual granulation time. Such quick simulations can also help improve control of 
continuous pharmaceutical processes. Modeling these processes also abides by 
Quality-by-Design(QbD) approach being employed by the industry. This work showcases the speed 
improvements that can be achieved using HPCs and how they can help improve process design.

The main objective of this work is to develop a uni-directional DEM-PBM model that runs efficinetly in 
parallel so that it can take advantage of the computational capabilities of  modern computing 
clusters. Specific studies performed in this work include:
\begin{itemize}
\item Development of 4-Dimensional, reduced order DEM informed PBM that is parallelized using hybrid 
techniques (MPI + OMP)to model the bulk processes occurring during the granulation process.
\item Perform DEM simulations on LAMMPS Improved for general granular and granular heat transfer 
simulations (LIGGGHTS) to  model the micro-mechanics of the L\"{o}dige CoriMix 
CM5 high shear granulator. 
\item Unidirectionally couple DEM to PBM using Radical Pilot, which is a framework developed in
Python which provides task level parallelization, to help develop an accurate model which can be 
run quickly on high performance computing systems
\end{itemize}

\section{Background \& Motivation}

\subsection{Particulate processes}
Particulate processes are ones in which a system of discrete species exist, such as granules 
or catalyst pellets, that undergo changes in average composition, size, or other pertinent 
properties. These processes are especially prevalent in the pharmaceutical industry and one of the 
most important is granulation. 
Fine powders are converted to larger granules using a liquid binder. The three processes 
influencing granulation are wetting and nucleation, consolidation and aggregation and attrition and 
breakage\citep{Iveson2001}\citep{Cameron2005}. As the liquid is added to the fine powder, it 
forms a porous nuclei that can coalesce, deform and break\citep{Barrasso2015ces}. There is an 
alteration in the properties of these nuclei as they can take up additional liquid or finer powder.
To understand how a granulation processes will behave with different parameter settings and 
chemical mixtures, experimental studies of the process parameter space are performed which is a 
method referred to as Quality-by-Testing (QbT). This methodology is time consuming and 
expensive. Thus, newer research is focused on the QbD concept i.e, 
mathematically modeling the process to better understand the dynamics of such processes.

\subsection{Modeling}
The paradigm shift of the pharmaceutical industry ofr instance towards continuous manufacturing, 
emphasizes the need for a more accurate model. This further helps develop better control strategies 
for the process. The modeling of particulate processes is more time consuming and computationally 
expensive when compared to a fluid systems since the particles are considered as individual entities 
rather than a continuum like in fluid systems. The models discussed ahead represent the 
particle-particle interaction at meso and micro-scale.
\subsubsection{Discrete Element Modeling (DEM)}
 Discrete Element Method is a simulation technique used to monitor the behavior of each particle 
as a separate entity compared to other bulk continuum models. This method tracks the movement of 
each particle within the space, records the collisions of each particle with the geometry as well 
as with each other and it is also subject to other force fields like gravity (\cite{Barrasso2015cerd}). This 
model is based on the Newton's laws of the motion and is expressed as in Equations 
\ref{eqn:bkgd_dem_n2law} and  \ref{eqn:bkgd_dem_forcebal} : \\
\begin{align}
m_i\frac{dv_i}{dt} &= F_{i,net} \label{eqn:bkgd_dem_n2law} \\
F_{i,net} &=  F_{i,coll} +  F_{i,ext} \label{eqn:bkgd_dem_forcebal}
\end{align}
In the above equations $m_i$ represents the mass of the particle, $v_i$ represents the velocity of 
the particle, $F_{i,net}$  represents the net force on the particle, forces on the particle due to collisions 
and other external forces are represented in $F_{i,coll}$ and $F_{i,ext}$ respectively.

The distance between each particle calculated at every time step and if the distance between two 
particles is less than the sum of the radii (for spherical particles)  a collision between the two particles 
is recorded. The tolerance for overlap is low in the normal as well as the tangential direction. 
Micro-scale DEM simulations are computationally demanding and simulations may take up to several 
days to replicate a few seconds of real time experiments. Many methods have been implemented to 
increase the speed of these simulations, such as scaling by increasing the size of the particles. These 
approximations are good in understanding the physics of the system but are not directly applicable to 
process-level simulations. 

The particle-particle collisions are not always elastic, thus there is a need for models for the 
contact forces. The earliest model was developed by \cite{hertz1882} was elastic in nature. This model was 
extended by \cite{mindlin1953} by accounting for the  tangential forces during the collisions. The dampening
in the velocity during the plastic deformation in a collision was accounted by \cite{Cundall1979} and 
\cite{walton1986}. \cite{tsuji1992} developed a more accurate model for predicting the behavior of granular flow.
The most commonly used contact force model for modelling the flow of particles inside the granulation process 
is the Hertz-Mindling contact model \citep{gantt2006}\citep{hassanpour2013}. 
 

\subsubsection{Population Balance Model (PBM)}
     Population balance models (PBM) predict how groups of discrete entities will behave on a 
    bulk scale due to certain effects acting on the population with respect to time 
    (\cite{ramkrishna2014}). In the context of process engineering and granulation, population 
    balance models are used to describe how the number densities, of different types of particles, in 
    the granulator change as rate processes such as aggregation and breakage reshape particles 
    (\cite{Barrasso2013}). A general form of population balance model is shown here as equation 
    \ref{eqn:bkgd_pbm_general}.
    
    \begin{align}
    \frac{ \partial}{\partial t}F(\textbf{v},\textbf{x},t) +& \frac{\partial}{\partial 
        \textbf{v}}[F(\textbf{v},\textbf{x},t)\frac{d\textbf{v}}{dt}(\textbf{v},\textbf{x},t)] + 
    \frac{\partial}{\partial \textbf{x}}[F(\textbf{v},\textbf{x},t)\frac{d\textbf{x}}{dt}(\textbf{v},\textbf{x},t)] 
    \notag\\
    &= 
    \Re_{formation}(\textbf{v},\textbf{x},t)+\Re_{depletion}(\textbf{v},\textbf{x},t)+\dot{F}_{in}(\textbf{v},
    \textbf{x},t)-\dot{F}_{out}(\textbf{v},\textbf{x},t)
    \label{eqn:bkgd_pbm_general} 
    \end{align}
    
In equation (\ref{eqn:bkgd_pbm_general}), $\textbf{v}$ is a vector of internal coordinates. For 
modeling a granulation process $\textbf{v}$ is commonly used to describe the solid, liquid, and gas 
content of each type of particle. The vector $\textbf{x}$ represents external coordinates, usually 
spatial variance. For a granulation process this  account for spatial variance in the particles as they 
flow along the granulator.

\subsubsection{Coupled DEM-PBMs}
The use of multi-physics models has recently been adapted to understand the behavior of 
particle systems. These models help understand the physics of the system at various scales 
\textit{i.e.} micro, meso and macro scale~\citep{sen2014}. Particle process dynamics have been 
inferred from coupling of various physics models \textit{viz.} CFD, 
DEM and PBM. Earlier works from \cite{sen2014} and \cite{Barrasso2015cerd} have successfully 
predicted process dynamics of the granulation process using such multi-physics models.

Initially, \cite{ingram2005} coupled PBM and DEM using two different multi-scale frameworks and 
focused on methods of integrating the two methods and information exchange required between the 
two. Later efforts on coupling of PBM and DEM were unidirectional in nature, where the collision 
data was obtained from the DEM and then used in PBM. \cite{gantt2006} used the DEM data 
to build a mechanistic model for the PBM. \cite{Goldschmidt2003} solved a PBM using DEM by 
replacing smaller particles as they successfully coalesce with larger particles. A mechanistic aggregation 
model was developed for wet granulation by \cite{Reinhold2012}\textbf{need to improve this sentence}. 
A hybrid model for one-way coupling has been reported for continuous mixing (\cite{sen2013} \&
\cite{sen2013b}) and is discussed in Section \ref{sec:pbm_model}.

In this work, a coupledof DEM and PBM model has been implemented. The PBM gives meso-scale 
information while the DEM gives particle scale information. The combination of these two methods helps 
describe the process dynamics with more accuracy. However, calculations involved due to the number 
of particles involved in the DEM process as well as PBM become very computationally heavy, hence the
motivation and need for . 
%The 
%recent development in the design of CPUs and increasing number of cores in the CPU, it makes 
%sense to utilize them to make the simulations faster. Thus, implementation of various parallel 
%computing techniques was employed in this work to help improve the simulation times. 

\subsection{Parallel computing and computer architectures}
%\subsubsection{Overview}
 The goal of parallel computing is to distribute large amounts of computation across many 
compute cores to solve problems faster (\cite{wilkinson2005}).
\subsubsection{Computer Architecture}
High Performance Computing clusters are a collection of nodes interconnected by a high speed 
communication network for message passing from one node to another. Analogous to a conventional 
PC, each node has one or more CPUs and RAM. Commonly nodes are manufactured with two CPUs, each CPU 
is a multi-core meaning it has multiple compute cores such that each can carry out calculations 
separately from one another. %CPUs also have built in memory called cache that is much faster than 
%RAM which is why for optimal performance cache utilization should be favored over RAM when possible. 
On a node, memory is divided by CPU sockets, so each CPU has direct access to memory that is local to 
its own socket, however accessing memory on another socket is much slower \cite{Jin2011}. 
For this reason. data that is needed for computation should be stored locally to the CPU that needs it.  

There are two classes of computer architecture which can be classified by memory locality features such as 
distributed memory systems or shared memory systems. These two classes co-exist in a cluster, 
t hus providing the benefits of each. All the nodes exchange memory using explicit message passing 
while each has its own independent memory. The cores on each node can access data from the 
shared memory without any explicit message passing statements from the user. While designing a 
parallel program all these aspects need to be considered for optimal performance of the code 
(\cite{Adhianto2007}).


\subsubsection{Parallel application programming interfaces}
Message Passing Interface (MPI) is a common parallel computing application programmig interface (API) 
standard. MPI is used for distributed memory parallel computing and this is because the API will operate 
every MPI process as a discrete unit that does not share memory with the other processes unless explicit 
message passing is used. Even on shared a single node where the hardware is supports shared 
memory computing, MPI will still operate it in a distributed memory fashion \citep{Jin2011}. Operating all 
cores as distinct units also means they each need their own copy of all variables used for computation 
which results in a large overall memory foot print compared to a similar system if it was operated in 
shared memory. 

Open Multi-Processing (OMP) is another application program interface stand for parallel 
computing. OMP is used for shared memory and can take advantage of shared memory systems which 
can result in much faster computation. Although, it does not work well on distributed systems. This 
prevents it from being used to efficiently carry out computations across multiple nodes of a cluster 
simultaneously \citep{Jin2011}. 

Since MPI is preferred for distributed computing and OMP is better for shared computing many 
individuals have studied the performance of MPI vs MPI+OMP methods and many studies have used 
MPI+OMP for scientific computation for improved performance. Often times a trade of is made 
between optimizing a program for performance and trying to make it flexible enough to run on many 
different computer architectures. In the conclusion to the work by \cite{Bettencourt2017} it was found 
that hybrid methods for PBMs allow the code flexibility for different architectures while still maintaining 
good performance.  It was also reported that only the external(spatial) coordinates of the PBM were 
parallelized. In this current work external and internal(compositions) calculations are parallelized. 

\subsubsection{Previous works on parallel development and solution of PBM and DEM}
The idea of parallelization to reduce the amount of time required has been employed by various 
researchers in the past. \cite{Gunawan2008} used high-resolution finite volumes solution methods for 
the parallelization of their PBM. They performed load balancing effectively by 
decomposing the internal coordinates of their PBM. They achieved speed improvements up to 100 
cores on one system size, but was not tested for models with more dimensions. Moreover, they 
mentioned that the parallelization could be improved using shared memory processing. 
\cite{Bettencourt2017} took a hybrid approach towards the parallelization of the PBM using both 
MPI and OMP. The hybrid parallelization helped achieve 
a speed improvement of about 98\% using 128 cores over the serial code. \cite{Prakash2013a} and \cite{Prakash2013b} 
used the inbuilt Parallel Computation Toolbox (PCT) in Matlab \citep{pctMatlab} to parallelize their PBM on lower number 
of cores, but this faced the shortcomings of Matlab's internal processing and could not achieve the 
speed improvements of parallelization of a program if it were written in a native programming language 
like C/C++ or FORTRAN. 

LAMMPS Improved for general granular and granular heat transfer simulations (LIGGGHTS) is an open-source software used to perform DEM simulations. This package natively 
supports MPI for parallelizing the simulation by static decomposition which partitions space such that 
the area of communication between the the MPI processes is minimized. \cite{kacianauskas2010} used 
load balancing methods similar to a static decomposition and observed that this works well for a 
mono-dispersed system but the computational effort increases for simulations for poly-dispersed 
material. \cite{Gopalakrishnan2013} also reported a speed increase and a parallel efficiency of about 
81\% in their CFD-DEM simulation. LIGGGHTS could not take advantage of shared memory interfaces 
since it did not support OMP. \cite{Berger2015} implemented hybrid parallelization methods for the 
particle-particle interaction and the particle-wall interaction modules in LIGGGHTS and also used the 
Zoltan library \citep{Boman2012} developed by Sandia National Laboratories for dynamic load 
balancing. They achieved a speed improvement of about 44\% for simulations performed on higher 
number of cores, but there was no significant speed improvement for smaller core counts. 

\subsection{Pilot abstraction and RADICAL-Pilot (RP)}
A primary challenge faced is the scalable execution of multiple (often two,
but possibly more) heterogeneous simulations that need to run independently
but have a need to communicate and exchange information. Traditionally each
simulation is submitted as an individual job, but that brings invariably leads
to the situation where each simulations gets through the batch-queue systems
independent of the other. So although the first-through-the-queue is ready to
run, it stalls fairly soon waiting for the other simulation to make it through
the queue.  On the other hand MPI capabilities can used to  execute both
simulations as part of a single multi-node job.  Thus whereas the former
method suffers from unpredictable queue time for each job, the latter is
suitable to execute tasks that are homogeneous and have no dependencies, and
relies on the fault tolerance of MPI which are inadequate.


The Pilot abstraction~\cite{review_pilotreview} solves these issues:  The
pilot abstraction, (i) uses a container-job as a placeholder to acquire
resources via the local resource management system (LRMS) and,  (ii) to
decouple the initial resource acquisition from task-to-resource assignment.
Once the pilot (container-job) is scheduled via the LRMS, it can then directly
be populated with the computational tasks. This functionality allows all the
computational tasks to be executed directly on the resources, without being
queued at the LRMS individually. This approach thus supports the requirements
of task-level parallelism and high-throughput as needed by science drivers.

RADICAL-Pilot is an implementation of the pilot abstraction, engineered to
support scalable and efficient launching of heterogeneous tasks
across different platforms.
\section{Methods}

\subsection{Discrete Element Modelling (DEM)}
\subsubsection{LIGGGHTS}
LAMMPS Improved for general granular and granular heat transfer simulations (LIGGGHTS) v3.60(\cite{Kloss2012}) developed by DCS computing was used for all the simulation performed 
in this study. Source code files were modified to obtain particle-particle 
collisions. The aforementioned version of LIGGGHTS was compiled using the mvapich (mvapich2 v2.1) 
and intel (intel v15.0.2) compilers with the -O3 optimization option as well as an option to side load the 
process to the Xeon phi co-processors was added. The initial timing studies were performed on Stampede 
supercomputer located at TACC, University of Texas, Austin. The hardware configuration of each node 
consists of 2 8-core Intel Xeon E5-2680 processors based on the Sandy Bridge architecture, 32 gb of 
memory with QPI interconnects at 8.0 GT/s PCI-e lanes.


\subsubsection{Geometry}    
% \textcolor{red}{check other file Charles uploaded to see if that one was more "journal ready"}

 In this study, the L\"{o}dige CoriMix CM5 continuous high shear granulator has been studied. Its 
geometry was developed using the SolidWorks$^{TM}$ (Dassault Syst\`{e}mes). This granulator 
consisted of a high speed rotating element enclosed within a horizontal cylindrical casing. The casing 
(shown in Figure \ref{fig:mthdsDemCharlesGranShell}) consists of a cylinder with diameter of 120 mm 
at the inlet and 130 mm at the outlet and having a total length of 440 mm. A vertical inlet port is 
provided at one end of the casing and an angled outlet port is provided at the larger end of the case. 

\begin{figure}
\centering
\includegraphics[scale=0.2]{shell_final_pic.pdf}
\caption{The isometric view of the L\"{o}dige CoriMix CM5 continuous high shear granulator casing.}
\label{fig:mthdsDemCharlesGranShell}
\end{figure}

The impeller consists of a cylindrical shaft of length 370 mm and diameter 68 mm with four 
flattened sides 15 mm wide running along the axis. The blades are placed on these flattened sides as 
shown in figure \ref{fig:mthds_dem_charles_impeller}. There are three different blade elements on the 
shaft (Figure \ref{fig:mthds_dem_charles_fig5pt3and4_blades_n_isometric}). At the granulator inlet, 
there are 4 paddle shaped feed elements following which there are 20 tear drop shaped shearing 
elements  and finally 4 trapezoidal blades near the exit. All these elements are placed in 
a spiral configuration. 
%The final configuration of the granulator is shown in Figure\ref{fig:mthds_dem_charles_fig5pt3and4_blades_n_isometric}.

\begin{figure}
\centering
\includegraphics[scale=0.15]{impeller_final_pic.pdf}
\caption{This shows the isometric view of the impeller inside the L\"{o}dige CoriMix CM5 continuous high shear granulator casing.}
\label{fig:mthds_dem_charles_impeller}
\end{figure}    

\begin{figure}
\begin{subfigure}{.3\textwidth}
\centering
\includegraphics[scale=0.075]{feed_element.pdf}	      
\caption{Feed element}
\label{fig:mthds_dem_feed_element}
\end{subfigure}%
\begin{subfigure}{.3\textwidth}
	\centering
	\includegraphics[scale=0.075]{exit_element.pdf}
	\caption{Exit element}
	\label{fig:mthds_dem_exit_element}
\end{subfigure}
\begin{subfigure}{.3\textwidth}
\centering
\includegraphics[scale=0.075]{shear_element.pdf}
\caption{Shear element}
\label{fig:mthds_dem_shear_element}
\end{subfigure}
\caption{Components (\subref{fig:mthds_dem_feed_element}) and (\subref{fig:mthds_dem_exit_element}) help in the forward 
movement of the particles while (\subref{fig:mthds_dem_shear_element}) aids to direct the particles to the wall 
inside the L\"{o}dige CoriMix CM5 continuous high shear granulator.}
\label{fig:mthds_dem_charles_fig5pt3and4_blades_n_isometric}
\end{figure}     




\subsubsection{Meshing}
 After the geometry was built in SolidWorks$^{TM}$ (Dassault Syst\`{e}mes) the shell and impeller 
were exported as stereolithography (STL) files. The coarsest output option was used to keep the STL files small and 
simple for faster computations times. They were also exported not keeping there original coordinates.  
This resulted in the impeller having 2802 faces and 1281 points with approximately a file size of 775 
kilobytes. The shell had 1948 faces and 720 points and size was about 544 kilobytes.  
 Meshlab was used to align the STL files for importing into LIGGGHTS. No mesh treatments were 
used on the STLs. 
 The meshes were then imported into LIGGGHTS using the write command in serial. This resulted 
in 50 elements of the impeller file having highly skewed elements, which have more than 5 
neighbors per surface or have an angle less than 0.0181185 degree, that according to LIGGGHTS 
would degrade parallel performance. The write exclusion list command in LIGGGHTS was used and 
this exclusion list file as then used in the simulation to skip the highly skewed elements during the simulation. 
%The shell did not have any skewed elements \textcolor{cyan}{(FUTURE SOLUTION? - perhaps we 
%can use the output from LIGGGHTS exclusion list to find exact elements of issue. then we can use 
%Meshlab to exclude those pieces or remesh those individual pieces into better shapes with less 
%skewed elements. might be better for a letter paper though}

\subsubsection{DEM input file settings}
The DEM simulation in LIGGGHTS are setup using an input script which defines the physical 
parameters of the particles, importing of the geometry, particle insertion commands, various physics 
models to be used during the simulation as well as various compute and dump commands to help print 
the data required for post-processing of the data. The particles were considered to be granular in 
nature. The Hertz-Mindlin model was used for non-adhesive elastic contact between the granular particles. 
The particles were inserted inside the granulator from the inlet at a constant mass flow rate of 15 
kilograms per hour. The rotation speed of the impeller was kept throughout the study at 2000 rotations 
per minute. Such a high rotation speed was chosen since this would lead to high shear between the 
particles and the walls of the shell resulting in better size control of the granules. There were 2 sets of 
simulations that were performed, one with mono-sized particles and second consisting of a distribution 
of sizes. The particle radii chosen for mono-sized simulation varied 0.59mm - 2mm, consecutive 
particles radii had volume twice of one before them. The radii range of the distributed size simulation 
was 1mm - 3mm. The difference in the mechanics of these two simulations is discussed in the Section \ref{sec:Results}. 
The physical constants used for the simulations are given in Table 
\ref{table:mthds_dem_input}.

The simulation data was collected after ever 50,000 time steps (5$\times 10^{-3}$ sec) 
for the visualization of the particles inside the shell, further post processing . The collisions 
between each of the particles and the collisions between of the particle and the geometry was 
collected and used in the PBM. 

\begin{table}
\caption{Physical Properties of the particle for the LIGGGHTS input script} 
\label{table:mthds_dem_input}
\begin{center}
\begin{tabular}{l|c|c}
\hline
\bf{Parameter} &\bf{Value} &\bf{Units}\\
\hline
Young's Modulus of particles  & $8 \times 10^{6}$ & $N.m^{-2}$\\
Young's Modulus of Geometry  & $1 \times 10^{9}$ & $N.m^{-2}$\\
Poisson's Ratio & $0.2$ & $-$\\
Coefficient of restitution (constant for all collisions) & $0.4$ & $-$\\
Coefficient of static friction & $0.5$ & $-$\\
Coefficient of rolling friction  & $0.2$ & $-$\\
Density of the granules & $500$ & $kg.m^{-3}$\\
\hline
\end{tabular}
\end{center}
\end{table}

\subsubsection{DEM data post processing}
The post processing of the data obtained from the DEM simulations was done using MATLAB. 
The first test run on the output data was to determine if the simulation had reached steady-state. The 
mass inside the granulator was found out by averaging it over 5 time steps and then compared to 
mass inside the granulator after every 10000 time steps (about 5$\times 10^{-4}$ sec) with a 
tolerance of about 10\%. If the mass was found to be constant for most of the iterations, it was 
considered to be at steady state. Another test to determine steady state was to monitor the number of 
collision inside the granulator. The visualization of the simulation data was done by running the 
LIGGGHTS post processing (LPP) script over the dump files to convert them into STL files. These 
files were then loaded in to Paraview~\citep{henderson2004} for a graphical representation of the 
simulation. It can be seen that the number of collision start to oscillate around a mean value. The 
number of collisions were then plotted and steady state time was determined.
A precautionary script was also run so as to determine that no particles were lost due to overlap 
of the geometry with the particles as well as from particle particle overlap.


\subsection{Population Balance Modelling (PBM)}
\label{sec:pbm_model}
\subsubsection{Model development}
 The population balance equation used in this work is expressed below:
\begin{align}
\frac{d}{dt}F(s_1,s_2,x,t)=\Re_{agg}(s_1,s_2,x,t)+\Re_{break}(s_1,s_2,x,t)+
\dot{F}_{in}(s_1,s_2,x,t)-\dot{F}_{out}(s_1,s_2,x,t)
\label{eqn:mthds_pbm_overall} 
\end{align}
where, $F(s_1,s_2,x)$ is the number of particles with an active pharmaceutical ingredients 
(API) volume of $s_1$ and an excipient 
volume of $s_2$ in the spatial compartment $x$. The rate of change of number of particles with time 
in different size classes depend on the rate of aggregation $\Re_{agg}(s_1,s_2,x)$ and the rate of 
breakage $\Re_{break}(s_1,s_2,x)$. Also, the rate of particles coming into, $\dot{F}_{in}(s_1,s_2,x)$ and 
going out, $\dot{F}_{out}(s_1,s_2,x)$ of the spatial compartment due to particle transfer affect the number of 
particles in different size classes. 
The rate of change of liquid volume for a given time in each particle is calculated using the equation: 

\begin{align}
\frac{d}{dt}F(s_1,s_2,x)l(s_1,s_2,x,t)&= 
\Re_{liq,agg}(s_1,s_2,x)+\Re_{liq,break}(s_1,s_2,x)+\dot{F}_{in}(s_1,s_2,x)l_{in}(s_1,s_2,x)\notag\\
&-\dot{F}_{out}(s_1,s_2,x)l_{out}(s_1,s_2,x)+F(s_1,s_2,x)\dot{l}_{add}(s_1,s_2,x)
\label{eqn:mthds_pbm_rate} 
\end{align}

where, $l(s_1,s_2,x)$ is the amount of liquid volume in each particle with API volume of $s_1$ and 
excipient volume of $s_2$ in the spatial compartment $x$. $\Re_{liq,agg}(s_1,s_2,x)$ and 
$\Re_{liq,break}(s_1,s_2,x)$ are respectively the rates of liquid transferred between size classes due to 
aggregation and breakage. $l_{in}(s_1,s_2,x)$ and $l_{out}(s_1,s_2,x)$ are respectively the liquid 
volumes of the particles coming in and going out of the spatial compartment. $l_{add}(s_1,s_2,x)$ is 
the volume of liquid acquired by each particle in the compartment at every time step due to external 
liquid addition.
Similarly, the rate of change of gas volume for a given time is calculated using the following equation: 

\begin{align}
\frac{d}{dt}F(s_1,s_2,x)g(s_1,s_2,x)&= 
\Re_{gas,agg}(s_1,s_2,x)+\Re_{gas,break}(s_1,s_2,x)+\dot{F}_{in}(s_1,s_2,x)g_{in}(s_1,s_2,x)\notag\\
&-\dot{F}_{out}(s_1,s_2,x)g_{out}(s_1,s_2,x)+F(s_1,s_2,x)\dot{g}_{cons}(s_1,s_2,x)
\label{eqn:mthds_pbm_gas_agg} 
\end{align}

where, $g(s_1,s_2,x)$ is the gas volume of each particle with API volume of $s_1$ and excipient 
volume of $s_2$ in the spatial compartment $x$. $\Re_{gas,agg}(s_1,s_2,x)$ and 
$\Re_{gas,break}(s_1,s_2,x)$ are respectively the rates of gas transferred between size classes due to 
aggregation and breakage. $g_{in}(s_1,s_2,x)$ and $g_{out}(s_1,s_2,x)$ are respectively the gas 
volume of the particles entering and leaving the spatial compartment. $\dot{g}_{cons}(s_1,s_2,x)$ is the 
volume of gas coming out of each particle at every time-step due to consolidation of the particles. 
The rate of aggregation, $\Re_{agg}(s_1,s_2,x)$ in Equation \ref{eqn:mthds_pbm_overall} is 
calculated as: \citep{Chaturbedi2017}

\begin{align}
\Re_{agg}(s_1,s_2,x)&= \frac{1}{2}\int _0^{s_1} \int_0^{s_2} 
\beta(s_1',s_2',s_1-s_1',s_2-s_2',x)F(s_1',s_2',x)F(s_1-s_1',s_2-s_2',x)ds_1'ds_2'\notag\\ 
&- F(s_1,s_2,x)\int _0^{s_{max_1}-s_1} \int_0^{s_{max_2}-s_2} 
\beta(s_1,s_2,s_1',s_2',x)F(s_1',s_2',x)ds_1'ds_2'
\end{align}


where, the aggregation kernel, $\beta(s_1,s_2, s_1',s_2',x)$ is expressed as a function of collision 
rate coefficient ($C$) and probability that collision results in agglomeration($\psi$) \citep{ingram2005}
and is shown below: 

\begin{align}
\beta(s_1,s_2,s_1',s_2',x) = \beta_oC(s_1,s_2,s_1',s_2',x)\psi(s_1,s_2,s_1',s_2',x)
%\beta(s_1,s_2,s_1',s_2',x) = & \beta_o*(V(s_1,s_2,x)+V(s_1',s_2',x))^{\gamma}*(c(s_1,s_2,x)\notag\\
%&+c(s_1',s_2',x))^{\alpha}\left(1-\frac{(c(s_1,s_2,x)+c(s_1',s_2',x))^{\delta}}{2}\right)^{\alpha}
\label{eqn:mthds_pbm_beta_kernal}
\end{align}

where, $\beta_o$ is aggregation rate constant.\\
Collision rate coefficient ($C$) is a function of particle sizes and is calculated by normalizing the 
number of collisions between group of particles \citep{gantt2006} and is obtained from LIGGGHTS 
DEM simulation. A recent study shows that collision frequency depends on PSD as well 
\citep{sen2014}. Collision rate coefficient for every time step can be expressed as:

\begin{align}
C(s_1,s_2,s_1',s_2')=\frac{N_{coll}(s_1,s_2,s_1',s_2')}{N_p(s_1,s_2)N_p(s_1',s_2')\Delta t}
\label{eqn:collfreq}
\end{align}

In Equation \ref{eqn:collfreq}, $N_{coll}$ is the number of collision between two particles in 
time interval $\Delta t$ \& $N_p$ is number of particle of particular size. The agglomeration 
($\psi$) in equation (\ref{eqn:mthds_pbm_beta_kernal}) can be expressed as:

\begin{align}
\psi((s_1,s_2,s_1',s_2') = 
\left\{\begin{matrix}
\psi_0,\hspace{0.2cm} LC((s_1,s_2) \geq LC_{min}\hspace{0.2cm} and\hspace{0.2cm} LC((s_1',s_2') \geq LC_{min}	\\ 
0,\hspace{0.2cm} LC((s_1,s_2) < LC_{min}\hspace{0.2cm} or\hspace{0.2cm} LC((s_1',s_2') < LC_{min}
\end{matrix}\right.
\label{eqn:colleff}
\end{align}
 In Equation \ref{eqn:colleff}, $LC$ is the liquid content of particles and $LC_{min}$ stands for minimum 
 liquid content required for coalescence of particles. 

%%%%% commented the breakage details as we aren't including in results


%	     Similarly, the breakage rate is expressed as-
%	
%	    \begin{align}
%	    \Re_{break}(s_1,s_2,x) = \int_0^{s_{max_1}} \int_0^{s_{max_2}} 
%K_{break}(s_1',s_2',x)F(s_1',s_2',x)ds_1'ds_2' - K_{break}(s_1,s_2,x)F(s_1,s_2,x)
%	    \end{align}
%%	     \textcolor{red}{citation?}
%	    
%	     where, the breakage kernel $K_{break}(s_1,s_2,x)$ is formulated as – 
%	    
%	    \begin{align}
%	    K_{break}(s_1,s_2,x) = C_{impact}\int_{U_{break}}^{\infty}p(U)dU
%	    
%K_{break}(s_1,s_2,x)=\left(\frac{4}{15\pi}\right)^{(\frac{1}{2})}G_{shear}exp\left(-\frac{B}{R(s_1,s_2,x)}\ri
%ght)
%	    \end{align}
%	     \textcolor{red}{citation?}

% where, $G_{shear}$ is the shear rate exerted by the impeller on the granules. $R(s_1,s_2,x)$ is 
%the radius of the granule that breaks and $B$ is the breakage kernel constant. $G_shear$ is 
%calculated as $\frac{\nu_{impeller}*D_{impeller}*PI}{60}$ where $\nu_{impeller}$ and $D_{impeller}$ 
%are respectively the rotational speed and diameter of the impeller.
The rate of increase of liquid volume of a particle, $\dot{l}_{add}(s_1,s_2,x)$ is expressed as:

\begin{align}
\dot{l}_{add}(s_1,s_2,x) = \frac{(s_1+s_2)(\dot{m}_{spray}(1-c_{binder})-\dot{m}_{evap})}{m_{solid}(x)}
\end{align}

where, $(s_1+s_2)$  is the total solid volume of the particle; $\dot{m}_spray$ is the rate of external 
liquid addition, $c_{binder}$ is the concentration of binder in the external liquid (which is assumed to 
be zero in this case); $\dot{m}_{evap}$ is the rate of evaporation of liquid from 
the system (which is also assumed to be zero in this case) and $m_{solid}$ is the total amount of solid 
present in the compartment.
The rate of decrease in gas volume per particle due to for a time step consolidation is calculated using the 
following expression: \citep{Verkoeijen2002} 

\begin{align}
\dot{g}_{cons}(s_1,s_2,x)=&c (\nu_{impeller})^{\omega}V(s_1,s_2,x)\frac{(1-\epsilon_{min})}{s} 
\notag \\ 
& \left[g(s_1,s_2,x)+l(s_1,s_2,x) -(s_1+s_2)\frac{\epsilon_{min}}{1-\epsilon_{min}}\right]
\end{align}        

 where, $c$ and $\omega$ are the consolidation constants; $v_{impeller}$ is the impeller 
rotational speed; $V(s_1,s_2,x)$ is the volume of particle, $\epsilon_{min}$ is the minimum porosity; 
$g(s_1,s_2,x)$ and $l(s_1,s_2,x)$ are respectively the gas and liquid volumes of the particle.

Particle transfer rate, $\dot{F}_{out}(s_1,s_2,x)$ in Equation \ref{eqn:mthds_pbm_overall} is calculated 
as:

\begin{align}
\dot{F}_{out}(s_1,s_2,x) = \dot{F}(s_1,s_2,x)\frac{\nu_{compartment}(x)*dt}{d_{compartment}}
\end{align}

where, $\nu_{compartment}(x)$ and $d_{compartment}$ are respectively the average velocity of 
particles in compartment $x$ and the distance between the mid-points of two adjacent compartment, 
which is the distance particles have to travel to move to the next spatial compartment. $dt$ is the 
time-step.
The process parameters and physical constants used in the PBM simulation are listed in Table 
\ref{table:mthds_pbm_parameters}.
\begin{table}
\caption{Parameters used in the PBM simulation}
\label{table:mthds_pbm_parameters}
\begin{center}
\begin{tabular}{l|c|c|c}
\hline
\bf{Parameter} &\bf{Symbol} &\bf{Value} &\bf{Units}\\
\hline
Initial time step & $\delta t$ & $0.5$ & $s$\\
Mixing time & $T$ & $25$ & $s$\\
Granulation time & $T$ & $75$ & $s$\\
Velocity in axial direction & $v_{axial}$ & $1$ & $ms^{-1}$\\
Velocity in radial direction & $v_{radial}$ & $1$ & $ms^{-1}$\\
Aggregation constant & $\beta_0$ & $1\times10^{-9}$ & $-$\\
Initial particle diameter & $R$ & $150$ & $\mu m$\\
%Breakage kernel constant & $B$ & $0$ & $-$\\
Diameter of impeller & $D$ & $0.114$ & $m$ \\
Impeller rotation speed & $RPM$ & $2000$ & $rpm$\\
Minimum granule porosity & $\epsilon_{min}$ & $0.2$ & $-$\\
Consolidation rate & $C$ & $0$ & $-$\\
Total starting particles in batch & $F_{initial}$ & $1 \times 10^{6}$ & $-$\\
Liquid to solid ratio & $L/S$ & $0.35$ & $-$ \\
Number of Compartments & $c$ & $16$ & $-$ \\
Number of first solid bins & $s$ & $16$ & $-$\\
Number of second solid bins & $ss$ & $16$ & $-$\\
\hline
\end{tabular}
\end{center}
\end{table}


\subsection{PBM Parallel \textit{C++}}
\subsubsection{Discretization \& parallelizing PBM}
The PBM was discretized by converting each of its coordinates in to discrete bins. For the spatial 
coordinates a linear bin spacing was used. For the internal coordinates, solid, liquid and gas a 
non-linear discretization was used \citep{barrasso2012}. 
Once the PBM had been discretized, a finite differences method was 
used which created a system of ordinary differential equations (ODEs) \citep{Barrasso2015cerd}. The 
numerical integration technique used to evaluate the system of ODEs was first order Euler integration 
as it is commonly used to solve these types of systems and author found an improvement in  speed while
having minimal impact on accuracy \citep{Barrasso2013}. In order to avoid numerical instability due 
to the explicit nature of the Euler integration, Courant-Friedrichs-Lewis (CFL) condition must be 
satisfied \citep{courant1967}. For our PBM model, time-step was calculated at each iteration such 
that, the number of particles leaving a particular bin at any time is less than the number of 
particles present at that time \citep{Ramachandran2010}. To obtain the most optimal parallel 
performance, when solving the PBM, work loads were distributed in a manner which took into account 
the shared and distributed memory aspects of the clusters, the PBM was being run on. To 
parallelize the model in a way which could take advantage of shared memory but still effectively run 
across a distributed system both MPI and OMP were implemented. One MPI process was used per CPU core and one OMP 
thread was used per CPU core, as \cite{Bettencourt2017} found it resulted in the best performance. 
MPI was used for message passing from one node to another while OMP was used for calculations on each 
node that could be efficiently solved using a shared memory system. 

An algorithm in the form of pseudo code is presented below illustrating how the calculations are distributed and 
carried out 
during the simulation. For each time step, the MPI processes are made responsible for a specific chunk 
of the spatial compartments. Then each OMP thread, inside of each MPI process, is allocated to one of 
the cores of multi-core CPU the MPI process is bound too. The OMP threads divide up and 
compute $\Re_{agg}$. After $\Re_{agg}$ is calculated the MPI 
processes calculate the new PSD value for their chunk at that specific time step, $F_{t,c}$. The slave 
processes send their $F_{t,c}$ to the master processes which collects them into the complete 
$F_{t,all}$. The master process then broadcasts the $F_{t,all}$ value to all slave processes. This 
decomposition of the data into different CPUs and further into various threads is illustrated in Figure 
\ref{fig:mthds_PBM_decompostion}.	

\begin{figure}
\centering
\includegraphics[scale=0.45]{PBM_decomposition.pdf}
\caption{The distribution of the PBM calculations to utilize the MPI + OMP parallelization technique on the CPUs of each node}
\label{fig:mthds_PBM_decompostion}
\end{figure}

A crucial feature of the PBM is that the current PSD ($F_{t,all}$) value is used to compute a new 
time step size for the next iteration. This means all of the MPI processes need to have the same 
dynamic time step size at each iteration for the calculations to be properly carried out in parallel. Since 
the completely updated $F_{t,all}$ value is shared before calculating a new time step each process will 
have the same $F_{t,all}$ value. As a result each process calculates the same size for the new time 
step. 

The PBM code was initially developed using a simple emperical kernel which did not take into account the DEM collision data.
The code utilized static arrays, whose size was allocated on initiation and then the code was parallelized using the same
technique as discussed above. This code was compiled using the intel/15.0.2 and mvapich2/2.1 compilers. The code was tested on
Stampede1 supercomputer for the configuration of the cores varying from the 1 to 128 cores. The speedup obtained for this simulation 
was about 37, i.e. the parallel code run using 128 processes was 37 times faster than the simulation which ran just on 1 core.
This was a good conformation for the hybrid parallel technique, thus being implemented in the coupled model. The results of these 
preliminary studies have been addressed in more detail in the supplemental information.


Since the DEM informed PBM code required dynamic size increase for the data structure due to different number of particles being 
present in the DEM simulation,
the PBM code developed used the Standard Template Library containers and their features available in \textit{C++} 11 and later. It 
requires intel v17.0.2 (gcc 5.3+)  or later for compilation. Since this module was not present on the Stampede, the execution of the 
PBM was done on the newer Stampede2 supercomputer. Each of the compute node of the cluster consists of Intel Xeon Phi 7250 
(\textquotedblleft Knights Landing\textquotedblright) which has 68 cores on a single socket clocked at 1.4 GHz, with 96 GB of DDR4 
RAM. Each of the cores consists of 4 hardware threads. The simulations were then run by varying the the number of MPI 
processes from 1 to 16 and the number of OMP threads from 1 to 8, thus, using a maximum of 128 cores.

\begin{algorithm}
\caption*{\textbf{Pseudo Code}}
\label{alg:MyAlgorithm}
\begin{algorithmic}[*]

\While{ $t<t_{final}$ }
\begin{itemize}[noitemsep,nolistsep,itemindent=10pt]
\item The spatial domain is divided into equal chunks)
\item Each MPI process is assigned on chunk of spatial domain shown as $c_{low}$ to $c_{up}$ 
\item Sum all $c_{low_i}$ to $c_{up_i}$ is = to [0,number of Compartments]
\end{itemize}
\For{each MPI processes} $c = c_{low_i}$ to $c_{up_i}$
\begin{itemize}[noitemsep,nolistsep,itemindent=20pt]
\item Each MPI process is further divided with OMP to take advantage of multi-core CPU
\item Each OMP process is allocated to a single compute core
\item $\Re$ integrates for each first solid bin as $(i_1)$ $\int_{0}^{s_{2(OMP_0)}}$, $(i_2)$ $\int_{s_{2(OMP_0)}}^{s_{2(OMP_1)}}$, $(i_3)$ 
$\int_{s_{2(OMP_1)}}^{s_{2(OMP_2)}}$   and so on are divided into smaller integrals  $\int_{i_1low_n}^{i_1up_n}$, $\int_{i_2low_n}
^{i_2up_n}$, $\int_{i_3low_n}^{i_3up_n}$ upto 16 bins 
which are solved by the "n" OMP processes. Figure \ref{fig:mthds_OMP_distribution} depicts this distribution.
\item Allocated to that MPI process (CPU)
\end{itemize}
\For{ each OMP process}
\begin{itemize}[noitemsep,nolistsep,itemindent=30pt]
\item Calculate the Aggregation Constant $\Re_{agg}(s_1,s_2,c)$
\end{itemize}
%\begin{align}
%\Re_{agg}(s_1,s_2,c)=& \frac{1}{2}\int_{0}^{s_{1}} \int_{i_1low_n}^{i_1up_n} 
%\beta(s_1',s_2',s_1-s_1',s_2-s_2',c)F(s_1',s_2',c)F(s_1-s_1',s_2-s_2',c)ds_1'ds_2'\notag\\ 
%&- F(s_1,s_2,c)\int _{0}^{s_{max_1}-s_1} \int_{i_2low_n}^{i_2up_n} 
%\beta(s_1,s_2,s_1',s_2',c)F(s_1',s_2',c)ds_1'ds_2'\notag
%\end{align}
%\begin{align*}
%\Re_{break}(s_1,s_2,c) = \int_{0}^{s_{max_1}} \int_{i_3low_n}^{i_3up_n} K_{break}(s_1',s_2',c)F(s_1',s_2',c)ds_1'ds_2' - K_{break}
%(s_1,s_2,c)F(s_1,s_2,c)
%\end{align*}
\EndFor
\begin{itemize}[noitemsep,nolistsep,itemindent=20pt]
\item  Calculate total number of particles using Euler's rule. 
\end{itemize}
%\begin{align*}
%F_{t,c} &= \frac{\Delta F(s_1,s_2,c)}{\Delta t}\Delta t  + F(s_1,s_2,c)_{t-1}\\
%      &=   (\Re_{agg}(s_1,s_2,c)+\Re_{break}(s_1,s_2,c)+\dot{F}_{in}(s_1,s_2,c)-\dot{F}_{out}(s_1,s_2,c))\Delta t + 
%F(s_1,s_2,c)_{t-1}
%\end{align*}

\EndFor
\begin{itemize}[noitemsep,nolistsep,itemindent=10pt]
\item MPI Send $F_{t,c}$ to Master MPI process
\item MPI Recv $F_{t,c}$ from MPI all slave processes
\item Master consolidate all $F_{t,c}$ chunks into a complete $F_{t,all}$
\item Master does inter-bin particle transfers (updates $F_{t,all}$)
\item MPI Bcast $F_{t,all}$ to all slave processes
\item $t_{new} = t + timestep$
\end{itemize}
\EndWhile   

\end{algorithmic}
\end{algorithm}   
\begin{figure}
\centering
\includegraphics[scale=0.5]{OMP_table.pdf}
\caption{The integration of aggregation rate distributed among OMP processes. This figure indicates the use of 4 OMP processes, the 
configuration that gave us the best performance. This technique has been implemented for each MPI process.}
\label{fig:mthds_OMP_distribution}
\end{figure}	

\subsection{Radical Pilot (RP) \& PBM+DEM communication}
\textbf{Giannis you need to add the paragraph on RP set up here!}



\section{Results \& Discussion}
\label{sec:Results}
\subsection{Discrete Element Modelling (DEM)}
\subsubsection{DEM Spatial Decomposition Studies}
LIGGGHTS as discussed previously, statically decomposes the work space and each section is sent to 
a MPI process for calculations. Thus, the division of the space becomes an important criteria for the 
simulation for efficient load balancing. The initial studies were undertaken for a mono-sized particle of 
size 1 mm and the simulation was carried out for 0.5 second of granulation time. The initial timing studies
for the decomposition were performed using 64 cores. The effect of the decomposition on the simulation 
time can be seen in Table \ref{table:rslts_dem_slicing_studies}. This indicates that dividing the 
x-direction in more number of compartments help increase the speed of the simulation. This is easy to 
comprehend since the granulator has its length parallel to the x-axis. These results also show that if 
the geometry is divided into more than 2 compartments in the y-axis or the z-axis the simulation time 
increases. This can be accounted to the increased communication required to transfer the rotating 
impeller mesh from one compartment in the y-axis or the z-axis to the another compartment for each 
time step. Since MPI is limited by communication in between the nodes, a speed decrease is observed 
due to increased partitioning in these directions.

\begin{table}
\caption{The effect of spatial decomposition on the performance of the DEM simulations using 64 cores.}
\label{table:rslts_dem_slicing_studies}
\begin{center}
\begin{tabulary}{\linewidth}{C|C|C|C}
  
\hline
\bf{Slices in x-direction}&\bf{Slices in y-direction}&\bf{Slices in z-direction}& \bf{Time taken for a 0.5 
second simulation (in minutes)}\\
\hline
$64$ & $1$ & $1$ & $10.27$\\
$32$ & $2$ & $1$ & $8.7$\\
$16$ & $2$ & $2$ & $6.83$\\
$8$ & $4$ & $2$ & $7.2$\\		  
$8$ & $2$ & $4$ & $7.96$\\
\hline  		  
\end{tabulary}
\end{center}
      
\end{table}

When MPI is used for parallelization of a task, load balancing becomes an important parameter 
that needs to be considered. When the geometry is divided, the amount of computation done by one core
should be in a similar to other processors. This helps in better utilization of the resources as well as
make the simulations run faster. Following the results from the initial timing studies,the y 
and the z-axes were not divided in more than 2 compartments for 128 and 256 core simulation as well. 
This meant that the x-axis was divided into 32 and 64 compartments respectively. In order to avoid the 
expensive communication between the processes, LIGGGHTS tries to insert the particles towards the 
center of the compartment such that the number of ghost atoms are minimized. But, slicing in the 
x-axis reduced the space available for the insertion of the particles thus, many of the particles were 
inserted incorrectly. Another abnormal behavior observed during these simulation was the particles 
halted at certain compartment and no particle traveled beyond this compartment in the x-direction. 
Thus, another set of timing studies were performed for the 128 and 256 core simulations. The 
comparison of simulation times have been shown in table .

\begin{table}
\caption{Comparison of time taken for the DEM simulations using 128 and 256 core due to different spatial decomposition 
configurations.}
\label{table:rslts_dem_128_256_decomp}
\begin{center}
\begin{tabulary}{\linewidth}{C|C|C|C|C}
  
\hline
\bf{Number of cores used}&\bf{Slices in x-direction}&\bf{Slices in y-direction}&\bf{Slices in 
z-direction}& \bf{Time taken for a 10 second simulation (in minutes)}\\
\hline
$128$ & $16$ & $2$ & $4$ & $264.67$\\
$128$ & $16$ & $4$ & $2$ & $247.2$\\
$256$ & $16$ & $2$ & $8$ & $271.5$\\		  
$256$ & $16$ & $8$ & $2$ & $252$\\
$256$ & $16$ & $4$ & $4$ & $265.32$\\
\hline  		  
\end{tabulary}
\end{center}
      
\end{table}
The simulation times illustrated in Table \ref{table:rslts_dem_128_256_decomp} show that 
incorrectly slicing the geometry also affects the performance of the system. It can be noted that slicing 
along y-axis is more favorable than slicing along the z-direction. The insertion of particles is hindered 
when the geometry is cut along the z-axis as the inlet is perpendicular to it. LIGGGHTS thus provisions 
lesser space for the insertion of these particles, thus increasing the time of the simulation. Thus, just 
slicing the geometry into 2 sections along the z-axis is preferred. So, the chosen configurations for the 
final simulations consisted of the geometry having only 2 sections in the z-direction and more slices 
along the y-axis. 

\subsubsection{DEM Performance}
A test case was run for mono-sized particle of 2mm diameter for timing comparison studies. The 
times are plotted in figure \ref{fig:rslts_DEM_2mm_timing} indicate that using lower number of cores is 
not feasible for long simulations since the time taken while using 1 core is about $11$x times slower.
Thus, the simulations were carried out in core configurations of 64, 128 and 256 cores. The studies 
undertaken had 5 mono-sized population of particles of diameter 0.63, 1, 1.26, 1.59 and 2 mm 
simulations and one simulation consisted of particle size distribution. Figure 
\ref{fig:rslts_DEM_timing_studies} shows that the amount of CPU time required for a 10 second 
simulation of the granulator. The post processing MATLAB script was run on the dump files obtained 
from the simulation and it was observed that the system reached a steady state about 3-5 seconds of 
the simulation time. Particles with larger diameter reached steady state at a faster rate with an average 
hold-up of particles of about 6500. 
\begin{figure}
\centering
\includegraphics[scale=0.75]{rslsts_2mm_timing_mtlb.pdf}
\caption{The variation in the amount of time taken for the DEM simulation of 2 mm particles as a function of number of cores. The 
speed increase on increasing the number of cores from 1 to 16 is higher than the speed improvements observed from increase of cores 
from 16 to 256.}
\label{fig:rslts_DEM_2mm_timing}
\end{figure}	

Pure timing studies are not really a good measure to represent the parallel performance of a program. 
Speedup of a parallel program indicates the speed increase of the program when it is run on more 
compute cores compared to the wall clock time when it is run in serial. It is the most common way to 
represent the parallel performance. Speedup is the ratio of the time taken to run the program in serial 
to the time taken by the program to run in parallel as shown in Equation \ref{eqn:rslts_DEM_Speed_Up}. 
For an ideally parallelized program, the speedup is 'n' times, where n is the number of cores used.

\begin{align}
\textit{Speedup} = \frac{\textit{Serial Wall Clock Time}}{\textit{Parallel Wall Clock Time}}
\label{eqn:rslts_DEM_Speed_Up}
\end{align}

Speedup does not take into account number of processors used in the simulation, thus another metric 
that is used to determine the parallel performance is parallel efficiency. This metric is nothing by 
speedup divided by the number of cores used. Thus, parallel efficiency normalizes speedup and gives 
a fractional value of the ideal speedup a program achieves with the increase in the number of cores.

\begin{align}
\textit{Parallel Efficiency} = \frac{\textit{Serial Wall Clock Time}}{\textit{Parallel Wall Clock Time $\times$ $n_{cores}$}}
\label{eqn:rslts_DEM_parallel_efficiency}
\end{align}

The speedup of the DEM simulations using only MPI cores is shown in Figure \ref{fig:rslts_DEM_speedup}.
There is a linear speed increase in the speedup upto 16 cores, after which the performance plateaus.
Figure \ref{fig:rslts_DEM_speedup} indicates that there 
is not a significant amount of speed improvement for the simulation when 256 cores are used for the 
simulation over 128 cores. This speed decrease can be accounted to the communication time between 
the MPI processes. When the particles move from one section to another of the space, they are 
transferred as ghost particles from one process to another process. Thus, there is large amount of 
communication which is required. One of the issues of using a cluster with shared memory within 
the nodes but none in between nodes is that it has to rely on the networking infrastructure 
of the cluster which bottlenecks the communications thus leading to higher communication times. 
There are more sections present when 256 cores are used for the 
simulation, thus there is more communication in this system when compared to 128 core simulation. 
This excess communication makes the simulation slower though there is more processing power and 
it requires lesser time for other calculations. Another observation that can be made from the Figure 
\ref{fig:rslts_DEM_timing_studies} is that the particle size distribution simulation takes more time 
compared to the simulations with mono-sized particles of 1.59mm and 2mm, though the mean size of 
particles in the distribution is 2mm. The default profile provided by LIGGGHTS indicates that the time 
spent in calculating the particle-particle interaction forces was higher than the mono-sized 
simulations. The different diameters make the interaction forces more tedious thus, making it 
computationally more expensive. 

\begin{figure}
\centering
\includegraphics[scale=0.75]{rslsts_2mm_DEM_speedup_mtlb.pdf}
\caption{The speedup achieved in the DEM simulations of 2 mm particles. There is a linear increase upto 16 cores, then as the parallel 
efficiency decreases, plateauing of the speedup is observed.}
\label{fig:rslts_DEM_speedup}
\end{figure}

\begin{figure}
\centering
\includegraphics[scale=0.85]{rslsts_DEM_alldia_timing_mtlb.pdf}
\caption{ Indicates the time taken for a 10 second DEM simulation for radii ranging from
 0.63mm to 2mm and a particle size distribution ranging from 1mm - 3mm. The speed improvements 
 for the smaller particles is higher as the core count is increased as compared to the larger particles.}
\label{fig:rslts_DEM_timing_studies}
\end{figure}
 The communication time in LIGGGHTS is indicated by the Modify time, which is the sum of the 
times required to transfer the rotating mesh from one process to the other. From Figure 
\ref{fig:rslts_DEM_percent_plot}, it can be seen the major chunk of the simulation time is taken up by 
the modify time. This is expected since the impeller is rotating at a very high speed of 2000 rpm. So, if 
the number of processes are increased the amount of time spent in transferring the mesh also 
increases. In the studies, the modify time as a percentage of the simulation increased from 82\% to 
about 90\%, when the core count was increased from the 64 cores to 256 cores. But, the using the 
higher number of cores reduces time taken to calculate particle-particle interaction as well as in 
neighbor calculation. Thus, a better implementation for meshing as well as decomposition of the 
geometry for faster simulations with higher core counts.

\begin{figure}
\centering
\includegraphics[scale=0.75]{rslsts_DEM_profle_mtlb.pdf}
\caption{The distribution of time taken by each component of the DEM simulation with varying number of cores for the 2mm particles. It 
can be observed that the percentage of time in communication between the processes is the highest due to the rotation of the impeller 
at high speeds.}
\label{fig:rslts_DEM_percent_plot}
\end{figure}


\subsection{Population Balance Modelling (PBM)}

%\subsection{PBM model validation without DEM data}
 
%\subsection{PBM model performance without DEM data}
% The PBM used for this study was initially tested without an DEM data using a constant value for the
%collisions. The PBM was tested on Stampede since it did not utilize the std libraries used to read
%and store the DEM data. This model was parallelized in the same manner as the mentioned above and 
%it achieved about 37\textit{x} speedup over the serial code. Table 
%\ref{table:rslts_performance_PBM_without_DEM} depicts the times of the experiments carried out 
%and performance improvements achieved.\\
%
%\begin{table}[ht]
%\caption{Performance studies of the PBM without the DEM data}
%\label{table:rslts_performance_DEM_without_DEM}
%\begin{center}
%\begin{tabulary}{\linewidth}{C|C|C|C|C|C}
%\hline
%\bf{Total cores in parallel}&\bf{MPI processes}&\bf{OMP threads}& \bf{Wall Time (seconds}& 
%\bf{Speedup}& \bf{Parallel efficiency}\\
%\hline
%$1$ & $1$ & $1$ & $222$ & $1.000$ & $1$\\
%$2$ & $1$ & $2$ & $147$ & $1.510$ & $0.755$\\
%$4$ & $1$ & $4$ & $109$ & $2.037$ & $0.509$\\
%$8$ & $1$ & $8$ & $103$ & $2.155$ & $0.269$\\		  
%$16$ & $2$ & $8$ & $41$ & $5.415$ & $0.338$\\
%$32$ & $4$ & $8$ & $21$ & $10.571$ & $0.330$\\
%$64$ & $8$ & $8$ & $10$ & $22.200$ & $0.347$\\
%$128$ & $16$ & $8$ & $6$ & $37.000$ & $0.289$\\
%\hline  		  
%\end{tabulary}
%\end{center}
%      
%\end{table}

\subsubsection{PBM model validation with DEM data}
 The population balance model implemented was considered to have an inlet flow of particles in the 
first compartment at a constant mass flow rate of 15 kilograms per hour. The particles were assumed 
to have a log normal distribution with minimum diameter of the particles being about 12 $\mu$m and the 
maximum of 2.1 mm with a mean of 0.453 mm and a standard deviation of about 0.62 mm. \\
The PBM used in this study employs an aggregation kernel that takes in account the formation of a 
larger particle from only two smaller particles. The ratio of the rate of formation and the rate of 
depletion due to aggregation helps us monitor whether the PBM satisfies the conservation of mass. 
Since this PBM takes into account the aggregation of only 2 particles at once, the ratio of the particles 
is expected to have a value of 0.5 during the aggregation process. Figure \ref{fig:rslts_PBM_ratio_plot_2mm}
 illustrates this ratio to be 0.5 throughout the simulation, validating the accuracy of the PBM used.
\begin{figure}
\begin{center}
\includegraphics[scale=0.5]{rslts_PBM_2mm_validation.pdf}
\caption{The ratio of rate of formation to the rate of depletion of particles due to 
aggregation as a function of time. The ratio is constant at 0.5 as aggregation is 
considered to happen only between 2 particles at a given instant of time.}
\label{fig:rslts_PBM_ratio_plot_2mm}
\end{center}
\end{figure}
Figure \ref{fig:rslts_PBM_d50_plots} shows four different particle size distribution plots at four 
different time instants. Figure \ref{fig:rslts_PBM_d50_plots}(\subref{fig:30s}) shows the distribution of 
the particles at 30 seconds, where we expect to have the highest number of the smaller particles. 
Since the degree of aggregation that has occurred is very low, thus there is a jump in the number of 
particles of the smaller size. The particle size distributions at 2 intermediate times of 50 seconds and 
75 seconds are plotted in Figures \ref{fig:rslts_PBM_d50_plots}(\subref{fig:50s}) and 
\ref{fig:rslts_PBM_d50_plots}(\subref{fig:75s}) respectively. These illustrate that there is an increase in 
the number of larger particles  and a subsequent decrease in the number of smaller particles. 
Figure \ref{fig:rslts_PBM_d50_plots}(\subref{fig:100s}) shows the distribution 
of the particle size at the end of the granulation process. It can be seen that the number of particles in 
the higher diameter bins increase.

\begin{figure}
\begin{subfigure}{.5\textwidth}
\includegraphics[scale=0.45]{rslts-PBM_30s_psd.pdf}
\caption{•}	
\label{fig:30s}
\end{subfigure}\hfill
\begin{subfigure}{.5\textwidth}

\includegraphics[scale=0.6]{rslts-PBM_50s_psd.pdf}
\caption{•}
\label{fig:50s}
\end{subfigure}
\begin{subfigure}{.5\textwidth}

\includegraphics[scale=0.45]{rslts-PBM_75s_psd.pdf}
\caption{•}
\label{fig:75s}
\end{subfigure}\hfill
\begin{subfigure}{.5\textwidth}

\includegraphics[scale=0.45]{rslts-PBM_100s_psd.pdf}
\caption{•}
\label{fig:100s}
\end{subfigure}
\caption{ Representation of the total number of particles inside all compartments with diameters less than 0.2 mm 
in mm after (\subref{fig:30s}) 30s (\subref{fig:50s}) 50s (\subref{fig:75s}) 75s and
 (\subref{fig:100s}) 100s of PBM simulation (there is only mixing taking place for the first 25 seconds).}
\label{fig:rslts_PBM_d50_plots}
\end{figure}   

\subsubsection{Parallel PBM performance studies}
 The PBM was run for the results obtained from each of the aforementioned DEM simulations. 
Since the PBM has been parallelized using hybrid techniques, a combination of MPI and OMP cores 
were used to perform the simulations. Figure \ref{fig:rslts_PBM_timing_studies} shows the average time 
taken by a PBM simulation for a total of 100 seconds of the granulation process, which includes 25 
seconds of the mixing time and 75 seconds of granulation time. The time taken for simulating all DEM 
scenarios by a single set of core configuration of in less than 10\% of each other, thus, an  average 
time for a single core configuration was used to illustrate the performance. Figure 
\ref{fig:rslts_PBM_timing_studies} shows that the program scales really well to about 32 cores but then, 
the improvement in the performance is negligible. The times represented have the core configurations 
of only MPI up till 16 cores after which 32, 64 and 128 core simulations employed 8 OMP threads 
each. The scaling with the only MPI cores shows substantial increase in performance. \\
\begin{figure}[H]
\centering
\includegraphics[scale=0.75]{rslsts_PBM_timing.pdf}
\caption{Average time taken to run the PBM simulation which consisted of 25 seconds of mixing time 
and 75 seconds of granulation time at different core configurations. There is a steady decrease as 
the number of MPI processes are increased, but the improvements on increasing the OMP threads are 
not that significant.}
\label{fig:rslts_PBM_timing_studies}
\end{figure}


Figure \ref{fig:rslts_PBM_speed_up} depicts the speed up achieved by the hybrid parallel PBM code. It 
can be seen when the MPI cores used are increased from 1 to 16 cores the speed up achieved is 
almost linear. This speed can be accounted to the way MPI has been implemented inside the code. 
Each compartment of the granulator is  offloaded on one MPI process, thus making 16 MPI processes 
the ideal since, the granulator has been divided into 16 compartments. When the number of cores 
were lesser than 16 cores, more than 1 compartment is sent to a single MPI process, this leads to a 
decrease in performance. The implementation of OMP on top of the MPI parallelization helps increase 
the performance by the about 10\%. The calculations inside the OMP parallel section of the code 
consists of large number of nested loop which have been known to be difficult to parallelize 
\citep{He2016} using the native \textit{C++}'s OMP libraries. The amount of communication time spent in 
between these threads is much higher than the speed increase achieved by using higher number of 
cores. One thread waits for another thread to complete processing the outer loops thus lower 
performance increase is achieved by using the OMP implementation. \\

\begin{figure}
\centering
\includegraphics[scale=0.75]{rslsts_PBM_speedup_logx.pdf}
\caption{The speedup achieved by the hybrid parallel implementation of the PBM program. It can be 
seen the initial speedup upto 16 cores as expected as from an ideal parallel program where as it 
becomes almost constant from 32 cores to 128 cores}
\label{fig:rslts_PBM_speed_up}
\end{figure}

A similar nature as above is observed in Figure \ref{fig:rslts_PBM_parallel_efficiency} where the parallel 
efficiency of the program has been plotted against the number of the cores used. This efficiency 
decreases as the number of cores used increase. The fall in efficiency initially up till 8 cores is steep as 
the communication time in between the MPI process decreases the efficiency, deviating its value from 
the ideal of 1 to about 0.4. But with the increase in the MPI processes to 16 and then the 
implementation of OMP the efficiency decreases as there is no major increase in the performance 
when compared to the increase in the number of cores used. The efficiency falls to as low as 11\% 
when 128 cores are used. Thus, there is scope for improving the OMP parallelizing the program 
especially where nested loops are present to decrease communication times.

\begin{figure}
\centering
\includegraphics[scale=0.75]{rslsts_PBM_efficiency.pdf}
\caption{The parallel efficiency for the hybrid (MPI + OMP) parallelized PBM code. The efficiency of the 
code decreases as the number of cores are increased. The higher number of cores have very low 
efficiency of about 11\% which depicts that there is large time being spent in communication in 
between the cores.}
\label{fig:rslts_PBM_parallel_efficiency}
\end{figure}

\subsection{DEM + PBM physics}
The micro-scale simulations provide an insight about the physics of the system usually by tracking 
each particle. This micro-scale simulation data is useful for the development of macro-scale 
mechanistic models which take into account the dynamics of bulk of the particles and not individual 
particles. A similar approach has been implemented in this work, where a mechanistic aggregation 
kernel was developed from the DEM particle-particle collisions. Thus, the aim of this section is to 
illustrate that the physics of the system does not change to a great extent with the change in the size 
of the particles or the distribution of the particles.

Two simulations from the current scenario will be compared, the DEM + PBM simulation of the 2mm 
mono-sized particle and the second being the simulation where the inserted particles were in a size 
distribution. Figure \ref{fig:rslts_PBM_psd_ratio} shows the ratio of rate of formation to the rate of 
depletion, both due to aggregation. The ratio obtained is 0.5 which indicates that the PBM is stable. 
Figure \ref{fig:rslts_PBM_psd_ratio} is also similar to Figure \ref{fig:rslts_PBM_ratio_plot_2mm}, thus 
indicating they have similar stability throughout the granulation process. 

One of the metrics to determine the physics of the system after a PBM simulation is to check the 
meidan diameter plots of the system after the granulation process. D\textsubscript{50} indicates the 
maximum diameter of particles that constitute 50\% of the total mass. These diameters vary along the 
length of the granulator since the granulated particles take time to pass through the granulator and 
that there is not enough liquid content in the later sections of the granulator to encourage the 
formation of granules. Thus, the d\textsubscript{50} for compartments in the latter section of the 
granulator is low. Figures \ref{fig:rslts_PBM_2mm_d50} and \ref{fig:rslts_PBM_psd_d50} show the 
d\textsubscript{50} plots of the mono-sized and psd respectively. It can be seen that both these plots 
have a similar behavior when it comes to the nature of the increase of the d\textsubscript{50} during 
the granulation process. There is a difference of about 20\% difference in the final diameter of the 
particles predicted, which at the of micrometers does not affect the final product quality. This slight 
deviation is observed due to the sudden jump in the rate of the aggregation which increases when 
particles from one compartment with higher d\textsubscript{50} get transferred to the next 
compartment with a lower d\textsubscript{50}. But, the advantage of running the mono-sized 
simulation compared to the psd simulation is the time taken to simulate the DEM. It can be seen from 
Figure \ref{fig:rslts_DEM_timing_studies} that the 2mm mono-sized particle took about 1.6 times less 
than the psd simulation time for the DEM. Since the physics of both the systems are not different, 
using the mono-sized simulations for the DEM help save time on the overall simulations. 

\begin{figure}
\centering
\includegraphics[scale=0.5]{rslts_pbm_psd_validation.pdf}
\caption{The ratio of rate of formation to the rate of depletion due to aggregation as a function of time. 
The constant value of 0.5 shows that the PBM is stable through the simulation time. This is similar to 
Figure \ref{fig:rslts_PBM_ratio_plot_2mm} which was for the 2mm mono-sized DEM simulation}
\label{fig:rslts_PBM_psd_ratio}
\end{figure}

\begin{figure}
\centering
\includegraphics[scale=0.5]{rslts_pbm_d50_128_200.pdf}
\caption{d\textsubscript{50} of the particles obtained after 100s of granulation time (25s of mixing and 75s of 
liquid addition) for the 2mm mono-sized particle DEM simulation.}
\label{fig:rslts_PBM_2mm_d50}
\end{figure}

\begin{figure}
\centering
\includegraphics[scale=0.5]{rslts_pbm_d50_128_555.pdf}
\caption{d\textsubscript{50} of the particles obtained after 100s of granulation time (25s of mixing and 75s of 
liquid addition) for the distributed particle size DEM simulation. The trend observed is similar to the trend 
found in Figure \ref{fig:rslts_PBM_2mm_d50}}
\label{fig:rslts_PBM_psd_d50}
\end{figure}


\subsection{One-way coupled DEM+PBM using RADICAL-Pilot and its performance} 
 The RADICAL pilot (RP) being a pilot job system it can bypass the need of waiting on the long scheduler
queues on a cluster. With resources allocated to itself the pilot can execute multiple simulation runs 
at once. The pilot job initiates the DEM simulation using the LIGGGHTS executable compiled for the DEM
studies and then creates a link between the collision data obtained, which is utilized by then utilized 
by the PBM executable. After the link is established, it also initiates the PBM. The timing profiles and
results from both the simulations are then obtained. Since the executables and inputs files used in the 
simulations and the RP job are the similar, we do not expect to see any changes in the physics of the 
system. A value of 0.5 was obtained for the ratio of the rate of formation to the rate of depletion due to aggregation
due to aggregation.
% which is confirmed from the plots of the ratio of the rate of formation to the rate of depletion 
%due to aggregation in Figure \ref{fig:rslts_RP_ratio_plot}
%\begin{figure}[H]
%\centering
%\includegraphics[scale=0.5]{rslts_PBM_2mm_validation.pdf}
%\caption{The ratio of rate of formation to the rate of depletion due to aggregation 
%as a function of time. Since the ratio is 0.5 as expected, we can say the physics does not change. 
%(\textbf{Currently a placeholder})}
%\label{fig:rslts_RP_ratio_plot}
%\end{figure}

%\subsubsection{RADICAL-Pilot performance}
 Since the platforms used for initial development and testing of the DEM and PBM respectively were different, 
the DEM simulations were executed again on Stampede2 for a fair and accurate comparison. The need for re-running
the simulations arises due to the clock speeds of the nodes and their configuration for the 2 platforms is different.
The RADICAL-pilot was setup for Stampede2 and aforementioned experiments were replicated. The DEM simulations were 
run for 64, 128 and 256 cores (MPI processes) and the PBM was run for 1, 2, 4, 8 and 16 MPI processes. No OMP threads 
were implemented for the PBM run, since the current version of RP does not support threading. 
 The times for the individual DEM and PBM simulations were added to determine the total amount time taken to 
simulate the one-way coupling. This time did not include the time spent by each executable in the queue of the 
scheduler. This time could vary from a few hours to a few days depending upon the load of the cluster. The total 
time taken for the simulation using the RADICAL-pilot was determined from the time the DEM was first executed until 
the PBM execution completed. Figure \ref{fig:rslts_RP_time_plot} shows a comparison of the total times of simulation 
for the DEM and PBM individual simulations to the simulation obtained using RP. It can be observed that the times taken 
by the pilot job are slightly higher than the summation of times of the individual simulations. The slight increase in 
the times of the simulation can be accounted to the communication required for the pilot realize that the DEM simulation 
has completed and link the data from the DEM to the PBM executable before it is executable. But, this communication is 
lower than the amount of time the executable the job would spend in the scheduler. Thus, the benefits of using RP become 
clear. An extension of this work would be to use RP to schedule multiple such coupled simulations as one job, thus 
decreasing the wait times even further. 
\begin{figure}
\centering
\includegraphics[scale=0.5]{rslts_RP_fake_plot.pdf}
\caption{A comparison of times taken by 2 individual simulations (DEM + PBM) (without adding queue time) to the  time taken by 
Radical-Pilot to complete the same set of simulation. A minimal increase in the time is observed, proving that using RP is 
a more convenient method to perform simulations which have large number of different submissions are required.
(\textbf{Currently a placeholder})}
\label{fig:rslts_RP_time_plot}
\end{figure}
	    
\section{Conclusions}
 A simple uni-directional DEM+PBM coupled study for a high shear granulator has been presented. DEM simulations were 
used to determine the movement of the particles inside the granulator and to obtain the collision data. This collision data 
was then used in the multi-dimensional PBM which was developed with 2 solid bins. Both these executables were parallelized 
to help them utilize the multi-core hardware of a cluster. The DEM was parallelized using MPI where as the developed PBM used 
MPI as well as OMP for a faster execution. The speed-up achieved for the DEM simulation was about 12 times using 64 cores, 
where as the PBM achieved a speed-up of 14 times when 16 cores were used. RP was utilized to execute the simulations for lower 
wait times.
 A more accurate model for the high shear granulator would be to couple the DEM and PBM bi-directionally i.e. these 
two methods are executed iteratively up till a steady state is reached. For the two-way coupling, the computation resources 
required would be really large as it would involve multiple DEM simulations. The role RP in such a situation would more crucial 
as the communication required would increase and it would help reduce total time taken to complete the multiple runs required. 

\section{Acknowledgements}
The authors woudl like to acknowledge National Science Foundation (NSF) for funding this project
through the grant number: 1547171.
\section*{References} 
\bibliographystyle{elsarticle-harv}
\bibliography{Bibli}
\end{document}
