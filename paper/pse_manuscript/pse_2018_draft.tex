% ==============================================================================================
% 2010-11-05 Preisig, H A       ; NTNU
% 2011-12-15  ditto             ; adapted for PSE 11
%                               ; put all the heading info into the style file
% 2012-01-04  ditto             ; Reference section title fixed -- integrated into style file
%                               ; changed sequence: first bibliography then pse style
%                               ;  pse style redefines reference listing superseding natlib def
%                               ;  and simplified by moving reference style loading into pse.sty
% 2014-09-04  ditto             ; reverted to standard LaTeX fonts
%                               ; issues - measures do not correspond with manual. 
%                               ; There is a cm missing 13.5 textwidth gives 12.5 texwidth ???
% 2015-10-14 Zinser, A          ; adapted to ESCAPE 26
% ==============================================================================================

\documentclass[fleqn,twoside,10pt]{article}

%% bibliography must be before escape styles
\usepackage{natbib}
\bibliographystyle{elsart-harv}

%% Escape/PSE styles
\usepackage{left_eq}   % To get the equation left aligned, this is ugly but required :-)
\usepackage{pse2018}  % Stylefile from Escape


%%% >>>>>>>>>>>>>>> use your own GRAPHICS configuration <<<<<<<<<<<<<<<<<<<<<<<
\usepackage{pstricks, pst-plot}	% pstricks
\usepackage{graphicx} % enables import of various formats
\usepackage{wrapfig}  % enables floating wrapped figures
\usepackage[figuresright]{rotating}
\usepackage{amsmath}
\usepackage[font=sf, labelfont={sf,bf}, margin=0cm]{caption}
\usepackage[inline]{enumitem}
\usepackage{epstopdf}   
\usepackage[margin=1in]{geometry}
\usepackage{lineno}
\usepackage{placeins}
\usepackage{subcaption}
%\usepackage[usenames,dvipsnames]{color}
%\usepackage[usenames,dvipsnames,svgnames,table]{xcolor}
\usepackage{algorithm}
\usepackage{algpseudocode}
\usepackage{textcomp}
\usepackage{graphicx}
\usepackage{tabulary}

%% math packages
\usepackage{amsmath}
\usepackage{amssymb}

%% >>>>>>>>>>>>>>>> user definitions BIBLIOGRAPHY & DEFS <<<<<<<<<<<<<<<<<<<<<<<<


%% personal styles and definitions
%\usepackage{mystyle}
%\def\mydefs{defs}
\usepackage[english]{babel}
\usepackage{blindtext}


% ================================================================================
\title{Parallel, multi-scale, mechanistic model for high shear granulation using
coupled DEM – PBM models on high performance computing systems.}
%
 \author[a]{Chaitanya Sampat}
 \author[a]{Yukteshwar Baranwal}
 \author[b]{Ioannis Paraskevakos}
 \author[b]{Shantenu Jha}
 \author[a]{Marianthi Ierapetritou}
 \author[a,*]{Rohit Ramachandran}
 \affil[a]{Department of Chemical and Biochemical Engineering, Rutgers University, Piscataway, NJ-08901, USA}
 \affil[b]{Electric and Computer Engineering, Rutgers University, Piscataway, NJ-08901, USA}
 \email{*rohitrr@soe.rutgers.edu}
 
 \HeaderTitle{Parallel, multi-scale, mechanistic model for high shear granulation using
coupled DEM – PBM models on high performance computing systems.}
 \HeaderAuthor{C. Sampat et al.}

% ==================================== title ====================================

\begin{document}

\maketitle             % make title page with abstract and keywords
\thispagestyle{empty}  % first page has a different format

% ================================================================================

\begin{abstract}
A multiscale model combines the computational efficiency of a macro-scale model and the
accuracy of a micro-scale model. It is preferred over a fully micro-scale model for its speed advantages while maintaining the physics 
of the problem. A less accurate way to perform such a simulation is touse data from a precomputed microscale model in a macroscale 
model. With the current cyberinfrastructure resources available, using more computationally intensive and concurrent multiscale models 
are morefeasible. This study proposes to use Discrete Element Method (DEM), a microscale model, and a Population Balance Model (PBM), 
a macroscale model, in a concurrent manner to model the granulation process of a pharmaceutical product inside a high shear 
granulator. The granulation between the components of a pharmaceutical blend is governed by the collision in between the particles. 
This leads to increase in their size, due to physical bonds in between them. The DEM provides the collision data while the PBM helps 
in predicting the macroscale phenomena like aggregation and breakage. This work attempts to couple these two models using a controller 
program, which triggers the DEM first, to give initial seed data to run the PBM. Then, the controller uses the data generated from the 
PBM continuously to determine the change in the physical properties and trigger the DEM from its last known state. The controller does 
the same with the DEM data to trigger the PBM. This occurs iteratively until a steady state is reached. A workflow diagram of the 
procedure followed is provided in Figure 1. The execution of each of the components is governed by a multilevel job scheduler which 
allocates resources rather than waiting for each simulation to run on a
normal job scheduler on a cluster. The DEM is parallelized using Message Parsing Interface (MPI) while the PBM is parallelized using a 
faster hybrid approach which is a combination of both MPI and Open Multi-Processing (OMP). Since the DEM is computationally heavy, an 
algorithm is developed to utilize the idle cores during the PBM execution to run multiple instances of the PBM such that parameter 
estimation of the kernels of the PBM occurs on the fly as well. This method of using shorter bursts of each simulation led to faster 
simulation times as well as a more accurate model of the high shear granulator. The Quality by Design (QbD) approach is addressed 
using such a modelling framework and it also helps us understand the
granulation process in a quantitative as well as in a mechanistic manner. \citep{Reference2015}
\end{abstract}
\Keywords{Multi-scale Model, Population Balance Model, Discrete Element Method, High-performance computing, MPI}

% ==================================== body  =====================================

\section{Introduction}
\begin{itemize}

\item Talk about pharmaceutical processes
\item Talk about DEM and PBM
\item Coupling and its importance
\item Advantages of using HPC
\end{itemize}

\section{Methods}
\subsection{Population Balance Model (PBM) development}
\begin{itemize}
\item Aggregation
\item Breakage
\end{itemize}
\subsection{Discrete Element Model (DEM) setup}
\subsection{PBM parallelization technique}
\subsection{Controller development}
Explain the controller configuration and how it works
\section{Results}
Discuss how many instances of the simulation of each occured
\subsection{Scaling results}
Cores used for DEM and PBM and how scaling occured
\subsection{Improved results over one way coupling}
Better d50 plots particle count
\section{Conclusion}

% ================================ references ===================================
%% References 

\bibliography{pse2018_psl}
\end{document}
% ================================== end =========================================