\documentclass[preprint,11pt,authoryear]{elsarticle}
\usepackage{amsmath}
\usepackage[font=sf, labelfont={sf,bf}, margin=0cm]{caption}
\usepackage[inline]{enumitem}
\usepackage{epstopdf}   
\usepackage[usenames,dvipsnames]{color}
\usepackage[usenames,dvipsnames,svgnames,table]{xcolor}
\usepackage{textcomp}
\usepackage{graphicx}
\usepackage{tabulary}
\usepackage{float}
\usepackage[utf8]{inputenc}
\usepackage{nomencl}
\usepackage[font=sf, labelfont={sf,bf}, margin=1cm]{caption}  % Margin similar to Anik
\usepackage[authoryear]{natbib}
\usepackage[margin=1in]{geometry}
\usepackage[english]{babel} 
\usepackage{multirow}
\usepackage{nomencl}
\makenomenclature

\usepackage[normalem]{ulem}
\makeatletter
\def\cyanuwave{\bgroup \markoverwith{\lower3.5\p@\hbox{\sixly \textcolor{cyan}{\char58}}}\ULon}
\def\reduwave{\bgroup \markoverwith{\lower3.5\p@\hbox{\sixly \textcolor{red}{\char58}}}\ULon}
\def\blueuwave{\bgroup \markoverwith{\lower3.5\p@\hbox{\sixly \textcolor{blue}{\char58}}}\ULon}
\font\sixly=lasy6 % does not re-load if already loaded, so no memory problem.
\makeatother

\newif\ifdraft
\drafttrue
\ifdraft
\usepackage{xcolor}
\definecolor{ocolor}{rgb}{1,0,0.4}
\newcommand{\terminology}[1]{ {\textcolor{red} {(Terminology used: \textbf{#1}) }}}
\newcommand{\jwave}[1]{ {\reduwave{#1}}}
\newcommand{\jhanote}[1]{ {\textcolor{red} { ***shantenu: #1 }}}
\newcommand{\csnote}[1]{ {\textcolor{blue} { ***chaitanya: #1 }}}
\definecolor{orange}{rgb}{1,.5,0}
\definecolor{dandelion}{cmyk}{0,0.29,0.84,0}
\newcommand{\gpnote}[1]{{\textcolor{green} {***giannis: #1}}}
\newcommand{\note}[1]{ {\textcolor{magenta} { ***Note: #1 }}}
\else
\newcommand{\terminology}[1]{}
\newcommand{\jwave}[1]{#1}
\newcommand{\jhanote}[1]{}
\newcommand{\csnote}[1]{}
\newcommand{\gpnote}[1]{}
\newcommand{\note}[1]{}
\fi



\journal{Chemical Engineering Research and Design}
\begin{document}
\begin{frontmatter}

\title{Parallel bidirectional coupling of DEM and PBM to model a high shear granulator using
Radical Pilot on HPCs.}
\author[add1]{Chaitanya Sampat}
\author[add1]{Yukteshwar Baranwal}
\author[add2]{Ioannis Paraskevakos}
\author[add2]{Shantenu Jha}
\author[add1]{Rohit Ramachandran\corref{cor2}}
\author[add1]{Marianthi Ierapetritou}
\address[add1]{Department of Chemical and Biochemical Engineering, Rutgers, The State University of New
Jersey, Piscataway, NJ, USA-08854}
\address[add2]{Electrical and Computer Engineering, Rutgers, The State University of New Jersey, 
Piscataway, NJ, USA-08854}
\cortext[cor2]{Corresponding author}
\ead{rohitrr@soe.rutgers.edu}

\begin{abstract}
A multiscale model combines the computational efficiency of a macro-scale model and the 
accuracy of a micro-scale model. It is preferred over a fully micro-scale model for its speed 
advantages while maintaining the physics of the problem. A less accurate way to perform such a
simulation is to use data from a precomputed microscale model in a macroscale model. With the
current cyberinfrastructure resources available, using more computationally intensive and 
concurrent multiscale models are more feasible This study proposes to use Discrete Element Method
(DEM), a microscale model, and a Population Balance Model (PBM), a macroscale model, in a
concurrent manner to model the granulation process of a pharmaceutical product inside a high
shear granulator. The granulation between the components of a pharmaceutical blend is governed
by the collision in between the particles. This leads to increase in their size, due to physical bonds
in between them. The DEM provides the collision data while the PBM helps in predicting the
macroscale phenomena like aggregation and breakage. This work attempts to couple these two
models using a controller program, which triggers the DEM first, to give initial seed data to run
the PBM. Then, the controller uses the data generated from the PBM continuously to determine
the change in the physical properties and trigger the DEM from its last known state. The controller
does the same with the DEM data to trigger the PBM. This occurs iteratively until a steady state
is reached. A workflow diagram of the procedure followed is provided in Figure 1. The execution
of each of the components is governed by a multilevel job scheduler which allocates resources
rather than waiting for each simulation to run on a normal job scheduler on a cluster. The DEM
is parallelized using Message Parsing Interface (MPI) while the PBM is parallelized using a faster
hybrid approach which is a combination of both MPI and Open Multi-Processing (OMP). Since
the DEM is computationally heavy, an algorithm is developed to utilize the idle cores during the
PBM execution to run multiple instances of the PBM such that parameter estimation of the kernels
of the PBM occurs on the fly as well. This method of using shorter bursts of each simulation led to
faster simulation times as well as a more accurate model of the high shear granulator. The Quality
by Design (QbD) approach is addressed using such a modeling framework and it also helps us
understand the granulation process in a quantitative as well as in a mechanistic manner. %\citep{barrasso2015cerd}
\end{abstract}
\begin{keyword}
Population balance model \sep Granulation \sep Discrete element methods  \sep MPI 
\sep Pharmaceutical process design
\end{keyword}
\end{frontmatter}
\section{Introduction}
\begin{itemize}
\item{Intorudce multi-scale models}
\item{Explain benefits}
\item{What is granulation \& how a multi-scale model helps with better modelling}
\item{Brief DEM and PBM works}
\item{Give an introduction to this work}
\end{itemize}

\section{Methods and Background}
\begin{itemize}
\item{Introduce particulate processes}
\item{Segway into DEM and PBM and why they are necessary}
\end{itemize}

\subsection{Discrte Element Method (DEM)}

\subsubsection{Background and previous works}

\subsubsection{DEM methodology used, Setup and Execution}
Talk about contact models used and other things

\subsection{Population Balance Model}

\subsubsection{Background and previous works}

\subsubsection{PBM Model Implemented}

\subsubsection{PBM discretization technique}

\subsection{Coupling of DEM and PBM}
Give a  background of the DEM and PBM coupling and why it is useful.
\subsubsection{Previous works}

\subsubsection{Controller of the PBM and DEM}

\subsubsection{Radical Pilot}
\begin{itemize}
\item{Background}
\item{Advantages}
\item{Setup}
\end{itemize}

\section{Results}

\subsection{Speed improvements and Scaling Studies}

\subsubsection{Speed studies of new MPI discretization}

\subsubsection{Speed improvements over one-way simulations}

\subsection{Process parameter study (DEM and PBM)}
\par This will have various sections depending upon the process parameters studied.
Some of them I can think of at the of my head
\begin{itemize}
\item{Impeller speed}
\item{Liquid Addition Rate}
\item{Material Density}
\end{itemize} 
\par Both these parameters will be changed in both the simulations, hopefully that results in a few changes.
May even have some timing profiles since that will affect the simulations.

\subsection{Radical Pilot Magic}
\par Show how radical pilot can used for multiple such couplings at once.
\textbf{Need to discuss with Giannis for complete details.}

\section{Conclusions}


\bibliographystyle{elsarticle-harv}
%\bibliographystyle{plain}
%\bibliography{ref_bidir}
\end{document}
